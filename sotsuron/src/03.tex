\chapter{システムモデル}\label{chap:system_model}
本章では,本研究で想定するシステムモデルを定義する.
まず,V2V通信を行う道路環境と車両配置について述べ,次にデュアルバンド構成の詳細を説明する.
続いて,ミリ波通信に影響を与える遮蔽モデルについて述べ,最後にシミュレーションで使用する通信パラメータを示す.

\section{伝搬モデル}
ローカル5G(L5G)では,免許申請時にカバーエリアおよび調整対象区域を算出するために,総務省が定めた伝搬モデルに基づく計算が求められる.
本節では,L5G免許申請支援マニュアル\cite{l5g_manual}に基づき,4.6--4.9GHz帯であるSub6での伝搬モデルについて述べる.

\subsection{受信電力の算出}
受信電力$P_r$は,以下の式で算出される:
\begin{equation}
    P_r = P_t + G_t - L_f + G_r - L - M
    \label{eq:received_power}
\end{equation}
ここで,$P_t$\,\si{[dBm]}は送信電力(基地局の空中線電力),$G_t$\,\si{[dBi]}は送信アンテナ利得,$L_f$\,\si{[dB]}は基地局の給電線損失,$G_r$\,\si{[dBi]}は受信アンテナ利得,$L$\,\si{[dB]}は伝搬損失,$M$\,\si{[dB]}はマージンである.マージンはSub6の場合8\,\si{dB}が規定されている.

\subsection{4.6--4.9GHz帯の伝搬損失モデル}
4.7GHz帯(Sub-6)における伝搬損失$L$は,自由空間伝搬損失式および拡張秦式を基礎として,基地局と陸上移動局間の距離$d_{xy}$\,\si{[km]}に応じて以下のように算出される.

\subsubsection{近距離($d_{xy} \leq 0.04$\,km)}
距離が40\,m以下の場合,自由空間伝搬損失式を用いる:
\begin{equation}
    L = L_0 = 32.4 + 20\log_{10}(f) + 10\log_{10}\left\{\frac{d_{xy}^2 + (H_b - H_m)^2}{10^6}\right\} + R
    \label{eq:l0_sub6}
\end{equation}
ここで,$f$\,\si{[MHz]}は使用周波数,$H_b$\,\si{[m]}は基地局の空中線地上高,$H_m$\,\si{[m]}は移動局の空中線地上高,$R$\,\si{[dB]}は建物侵入損(屋内設置の場合16.2\,\si{dB})である.

\subsubsection{中距離($0.04 < d_{xy} < 0.1$\,km)}
距離が40\,mを超え100\,m未満の場合,近距離モデルと遠距離モデルの補間を行う:
\begin{equation}
    L = L_0 + \{2.51 \times \log_{10}(d_{xy}) + 3.51\} \times (L_H - L_0)
    \label{eq:interpolation}
\end{equation}

\subsubsection{遠距離($d_{xy} \geq 0.1$\,km)}
距離が100\,m以上の場合,拡張秦式を用いる:
\begin{align}
    L = L_H = & \, 46.3 + 33.9\log_{10}(2000) + 10\log_{10}\left(\frac{f}{2000}\right) \nonumber     \\
              & - 13.82\log_{10}(\max(30, H_b)) \nonumber                                            \\
              & + \{44.9 - 6.55\log_{10}(\max(30, H_b))\} \cdot (\log_{10}(d_{xy}))^\alpha \nonumber \\
              & - a(H_m) - b(H_b) + R - K - S
    \label{eq:extended_hata}
\end{align}
ここで,$\alpha$は遠距離に対する係数,$a(H_m)$は移動局高に対する補正項,$b(H_b)$は基地局高に対する補正項である.
$S$は環境による補正値であり,市街地では0.0\,\si{dB},郊外地では12.3\,\si{dB},開放地では32.5\,\si{dB}が適用される.



