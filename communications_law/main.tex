\documentclass[10pt, twocolumn, a4j, uplatex]{jsarticle}	% Japanese
\usepackage[subrefformat=parens]{subcaption}
\usepackage{enumerate}
\usepackage{booktabs}
\usepackage[dvipdfmx]{graphicx}
\usepackage[dvipdfmx]{color}
\usepackage{amsmath}
\usepackage[varg]{txfonts}
\usepackage{comment}
\usepackage{float}
\usepackage{here}
\usepackage{threeparttable}
\usepackage{caption}
\usepackage{vtable,hhline}
\captionsetup[table]{justification=centering}
\bibliographystyle{IEEEtran}
\pagestyle{empty} %ページ番号非表示
\renewcommand{\baselinestretch}{0.68}	%ストレッチ
\renewcommand{\rmdefault}{ptm}			%Timesフォント
\setlength{\textwidth}{186mm}			%本文全幅    210-1inch(25.4)*2-186+13.4*2 = 0
\setlength{\textheight}{276.2mm}		%本文全高    297-1inch(25.4)*2-276.2+15*2 = 0
\setlength{\oddsidemargin}{-13.4mm}		%用紙左余白  210-1inch(25.4)*2-186+13.4*2 = 0
\setlength{\evensidemargin}{-13.4mm}	%用紙左余白  210-1inch(25.4)*2-186+13.4*2 = 0
\setlength{\topmargin}{-15mm}			%用紙上の余白 297-1inch(25.4)*2-276.2+15*2 = 0
\setlength{\headheight}{0mm}			%ヘッダーの高さ
\setlength{\headsep}{0mm}			    %ヘッダーと本文までの高さ
\setlength{\footskip}{0mm}			    %フッターと本文の高さ
\setlength{\textfloatsep}{3mm}			%本文と図の間のスペース
% \usepackage[acronym,shortcuts]{glossaries}
\setlength\intextsep{1pt}
\setlength\textfloatsep{8pt}
\usepackage{url}
\newcommand{\argmax}{\mathop{\rm arg~max}\limits}
\newcommand{\argmin}{\mathop{\rm arg~min}\limits}
% 参照系
\newcommand{\wpage}[1]{\pageref{#1}ページ}%
\newcommand{\weq}[1]{式(\ref{eq:#1})}%
\newcommand{\wfig}[1]{図\ref{fig:#1}}%
\newcommand{\wtab}[1]{表\ref{tab:#1}}%
\newcommand{\wpro}[1]{プログラム\ref{pro:#1}}%
\newcommand{\wsec}[1]{\ref{sec:#1}}%
\newcommand{\wpeq}[1]{\wpage{eq:#1}\weq{#1}}%
\newcommand{\wpfig}[1]{\wpage{fig:#1}\wfig{#1}}%
\newcommand{\wptab}[1]{\wpage{tab:#1}\wtab{#1}}%
\newcommand{\wppro}[1]{\wpage{pro:#1}\wpro{#1}}%
\newcommand{\wpsec}[1]{\wpage{sec:#1}\wsec{#1}}%
\newcommand{\wsubfig}[2]{\wfig{#1}\subref{fig:#2}}%\wsubfig{caption}{subcaption}
\begin{document}
\makeatletter
\def\section{\@startsection {section}{1}{\z@}{3 ex}{1ex}{\large\bf}}
\def\subsection{\@startsection {subsection}{1}{\z@}{3ex}{1ex}{\normalsize\bf}}
\def\subsubsection{\@startsection {subsubsection}{1}{\z@}{1ex}{1ex}{\normalsize\bf}}
\makeatother

\section{チェックポイント1}
\subsection{ITUとはどのような機関ですか?}
ITU(International Telecommunication Union:国際電気通信連合)は国際連合の専門組織のひとつで,本部がジュネーブにある電気通信に関する事柄を扱う国際機関です.
1932年に「万国電信連合」と「国際無線電信連合」が統合してITUとなり,1947年に国連の専門機関になりました.
加盟の枠組みとして,構成国193か国に加え,構成国が認める事業体・学術団体などの部門構成員,準構成員,教育機関といった参加形態があります.
組織面では,全権委員会議(PP)・理事会(C)などの会議体のもとに,ITU-T(標準化)、ITU-R(無線通信)、ITU-D(開発)といった部門を持つ構成になっています.

\subsubsection*{引用}
\begin{itemize}
  \item 資料2.pdf p.9(ITUの位置づけ/本部/構成員/組織)
  \item レジメ1.pdf(通信法規\_01)p.6(1932年の統合,1947年に国連専門機関)
\end{itemize}

\subsection{ITUの基本文書にはどのようなものがありますか?}
\subsubsection{国際電気通信連合憲章 (Constitution)}
前文,9章58条の規則,附属書からなる基本文書である.
内容としては連合の目的,構成・組織,電気通信に関する原則的規定,および条約の改廃などの基本的事項を定める.

\subsubsection{国際電気通信連合条約 (Convention)}
6章42条の規則,附属書からなる基本文書である.
内容としては,憲章を補足する内容になっており,具体的には連合の運営,総会及び所会議,財務,通信料金の国際精算に係わる原則,相互通信に係わる原則,暗語の使用,紛争の仲裁,条約の改正などである.つまり,連合の運用に係わる事項を規定している.

\subsubsection{国際電気通信規則 (ITR)}
連合の業務規則のひとつであり,憲章第1条に掲げる連合の目的を達成するために,電気通信事業に係わる事項を規定する.
具体的には,国際間における電気通信業務の提供・運用,料金の設定・決済や国際間の電気通信業務に使用される基盤的な電気通信手段に関する一般規則,主管庁ならびに認められた私企業に適用する規則などを規定している.

\subsubsection{無線通信規則 (RR)}
電波を「人工的導波体のない空間を伝搬する当面3000GHzより低い周波数の電磁波」と定義している.
空間を伝搬する電波は本質的に混信の可能性を持ち,混信が一定限度を超えると通信が不可能になるため,主要テーマの下で無線通信に係る事項を規定している.
主要テーマには,希少かつ有限な電波の有効利用,より多くの周波数資源を利用可能にする技術開発,円滑通信確保のための混信の制御である.
より具体的には周波数スペクトル・静止衛星軌道の公平かつ合理的な利用の促進,遭難・安全通信の周波数を有害な混信から保護し利用を保証,異なる主管庁間の無線通信の有害混信の防止と解決の援助,すべての無線通信の効率的・効果的運用の促進,新たな無線通信技術の応用を提供し,必要に応じて規律などがある.
例として分配と割り当ての違いが規定されており,分配(Allocation)とは周波数分配表で,特定の無線業務等に使うため周波数帯を指定すること(RR1.16)であり,割当て(Assignment)とは管庁が,局に無線周波数やチャンネルの使用許可を与えること(RR1.18)である.有害混信を避けるための分配・(周波数の割り振り)・割当て登録を行う,という憲章第1条がある.
\subsubsection*{引用}
\begin{itemize}
  \item 資料1.pdf:6ページ(憲章の構成・規定内容,前文の趣旨)
  \item 資料1.pdf:8ページ(条約の構成・規定内容)
  \item 資料2.pdf:13ページ(RRの定義・主要テーマ・基本項目)
  \item レジメ2.pdf:1〜4ページ相当(学習ガイド:分配・割当ての国際ルール,用語定義など)
  \item 資料1.pdf(ITRの説明箇所)(ITRが業務規則であること/3項目)
\end{itemize}

\subsection{「国際電気通信業務を利用する公衆の権利」についてITU憲章に規定するところを述べなさい}
ITU憲章(第6章・第33条「国際電気通信業務を利用する公衆の権利」)では,構成国(Member States)は公衆に対し,国際公衆通信業務によって通信する権利を承認すると規定しています.
また,その業務(サービス提供)・料金・保証(保護/保障)は,すべての利用者に対して,いかなる優先権または特恵も与えることなく同一とする,という「無差別(同等取扱い)」も定めています.

\subsubsection*{引用}
\begin{itemize}
  \item 資料1.pdf p.7「電気通信に関する一般規定(憲章 第6章 第33条から第41条)」の第33条要
  \item 資料2.pdf p.11 同趣旨の表(第33条の要約)
  \item レジメ1.pdf p.18(通信法規\_01 p.18)第33条の要約
\end{itemize}

\subsection{「人命の安全に関する電気通信の優先順位」についてITU憲章に規定するところを述べなさい}
ITU憲章では,「人命の安全に関する電気通信」について 絶対的優先順位 を与えることが規定されています.すなわち,構成国は,人命の安全に関するすべての電気通信に加えて,WHO(世界保健機関)の伝染病に関する特別に緊急な電気通信についても,他の通信に先立って取り扱う義務があります(憲章 第40条)。
また,無線通信に関しては,遭難通信の取扱いとして,無線通信の局は遭難の呼出し・通報を(どこから発せられたかを問わず)絶対的優先順位で受信し,応答し,直ちに必要な措置をとる義務があると定められています(憲章 第46条)。
同じ並びで,官用電気通信についても「利用者の要求がある場合には,可能な範囲で優先的に取り扱う」旨が規定されています(憲章 第41条)。

\subsubsection*{引用}
\begin{itemize}
  \item レジメ1.pdf:p.18(憲章 第40条)、p.19(憲章 第46条)、p.18(憲章 第41条)
\end{itemize}

\subsection{「無線通信業務」と「無線局の種別」の間にはどのような関係がありますか?}
「無線通信業務」と「無線局の種別」の関係は,“業務(サービス)を,具体的な無線局(ステーション)として実現する”という対応関係です.
無線通信業務(=業務 / Services)は「電波の使用目的」による分類で,無線通信業務はさらに個別の業務に分類されます.して,その業務を具体的に実現するものが「局」(無線局)です.
無線局の種別は,まさに「恒久的または一時的に運用する業務による局の分類」であり,どの業務を行う局かで種別が決まります.
この対応は,無線通信規則(RR)第1条の「業務と局の種別」の表として整理されていて,例えば,
\begin{itemize}
  \item 固定業務 ↔ 固定局
  \item 陸上移動業務 ↔ 移動局・基地局(など)
  \item 海上移動業務 ↔ 海岸局・船舶局(など)
  \item 航空移動業務 ↔ 航空局・航空機局
\end{itemize}
のように,業務ごとに用いる無線局の種別が対応付けられています.
さらに日本の免許手続でも,免許申請は「無線局の種別」に従って行う(種別が制度上の“入口”になる)と整理されています.

\subsubsection*{引用}
\begin{itemize}
  \item レジメ2.pdf(通信法規\_02)7〜8ページ:業務(Services)→局,局の種別$=$業務による局の分類,RRの対応表
  \item 資料2.pdf 15ページ:RR第1条「業務と局の種別」対応表・説明
  \item 通信法規\_04(レジメ4相当)3ページ:免許申請は「無線局の種別に従い…」
\end{itemize}

\subsection{「周波数帯の分配」と「周波数の割当て」の違いは何ですか?}
「周波数帯の分配」と「周波数の割当て」は,“何を(帯域か/個別周波数か)・誰に(業務か/局か)決めるか”が違います.
RR4.4 では,周波数分配表やRRの規定に反して周波数を局に割当ててはならないのが原則で,例外は「有害な混信を与えない」「保護を要求しない」などを明示条件とする場合です.つまり,分配(帯域→業務)で決まった枠の中で,割当て(局→個別周波数)をするのが基本です.波数割当て(新規・変更)はRR第Ⅲ章の手続(調整・通告・登録など)の対象で,ITUの**国際周波数登録原簿(MIFR)**に記録されることで,国際的に認知され権利主張が可能になる,と説明されています.フロー(事前公表→国際調整→通告→審査→登録)も示されています.
\subsubsection{周波数帯の分配(Allocation)}
一定の周波数帯を1つ以上の無線通信“業務”(例:固定業務,移動業務など)に使うために,周波数分配表(Table)で指定することです.さらに,指定した周波数帯を細分化して指定することも含みます.
\subsubsection{周波数の割当て(Assignment)}
ある“局”が,特定の条件の下で無線周波数またはチャンネルを使うことについて,主管庁(Administration)が与える許可です.

\subsubsection*{引用}
\begin{itemize}
  \item 資料2.pdf p.17(「分配(Allocation)」「割当て(Assignment)」の定義:RR1.16 / RR1.18)
  \item 資料2.pdf p.21(RR4.4:分配表等に反する割当ての禁止と例外条件)
  \item 資料2.pdf p.20(周波数割当て手続・MIFR・割当てフロー)
\end{itemize}

\subsection{周波数帯の分配における「一次業務」と「二次業務」の違いは何ですか?}
周波数帯を分配(周波数分配表で「この周波数帯はこの業務に使ってよい」と決める)するとき,同一の周波数帯を2つ以上の業務に分配する場合があります.このとき「有害な混信からどの程度保護されるか」で,一次業務と二次業務に区分されます.

\subsubsection{一次業務}
有害な混信からの保護を要求できる(=保護を受ける側として優先される)。
そのため,同一帯域内では(原則として)二次業務より優先します.

\subsubsection{二次業務}
一次業務の局に有害な混信を与えてはならない(一次業務を守る義務)。
一次業務の局からの有害な混信に対して保護を要求できない(一次業務が優先なので「守って」と言えない)。
ただし,後日割当てられる同一または他の二次業務の局からの有害な混信については,保護を要求できる(二次業務どうしでは先に割当てられたほうが一定の権利を持つ)。

\subsubsection*{引用}
\begin{itemize}
  \item レジメ2.pdf(通信法規\_02)p.15「一次業務と二次業務(RR 5.23からRR 5.33)」
  \item レジメ2.pdf(通信法規\_02)p.16 冒頭「(RR 5.30)(RR 5.31)」
\end{itemize}

\subsection{RR で定める「混信」にはどのようなものがありますか? \\
  また,電波を利用するうえで特に問題とされる混信は何ですか?}
RR(無線通信規則)では,まず「混信」を**“発射・輻射・誘導(またはその組合せ)による不要なエネルギーのために,受信時に情報の品質低下・歪・消失として現れる影響”**と定義しています(RR1.166)。
\subsubsection{RRで定める「混信」の種類}
影響(深刻さ)による分類(RR1.167から1.169)は以下の通りである.
\begin{itemize}
  \item 有害な混信(RR1.169):無線航行業務など安全業務の機能を害する,またはRRに従って行われる無線通信業務の運用を著しく低下・妨害・反復的に中断する混信.
  \item 許容し得る混信(RR1.167):RR/ITU-R勧告/RR上の特別協定で定める混信レベルや共用基準を満たす(観測または予測される)混信.
  \item 容認した混信(RR1.168):許容し得る混信レベルを超えるが,2以上の関係主管庁間で(他の主管庁を害することなく)合意された混信.
\end{itemize}

発生源による分類(RR第15条)は次の通りである.
\begin{itemize}
  \item 無線局からの混信(RR15.1から15.11)
  \item 電気機器・電気設備からの混信(RR15.12から)
  \item 産業科学医療(ISM)機器からの混信(RR15.13から)
\end{itemize}

電波利用上,特に問題とされる混信は「有害な混信」です.資料でも,円滑な無線通信の実施の面で制御が問題となるのは「有害な混信」であり,その防止について特に多くの規定が設けられていると明記されています.またRRの一般原則としても,有害な混信を生じさせないための対応や,国際遭難周波数・国際非常用周波数の保護が挙げられています.

\subsubsection*{引用}
\begin{itemize}
  \item 資料2.pdf:p.12(「混信」の定義RR1.166,有害/許容/容認の混信,どれが問題か,発生源分類)
  \item レジメ2.pdf:p.15(一般原則:有害な混信を生じさせない対応,遭難/非常用周波数の保護)
  \item レジメ2.pdf:p.17(影響面の分類+発生源の分類,RR15.22から15.46等)
\end{itemize}

\subsection{電波を発射する無線局(送信局)の設置について,RR にはどのような規定がありますか?}
RR(無線通信規則)では,電波を発射する無線局(送信局)を「設置(established)」または「運用(operated)」するための大前提として,国(主管庁)による許可(ライセンス)が必要だと規定しています.

送信局の設置・運用には「国の許可書(licence)」が必要である.「私人または企業は,当該局が属する国政府(またはその代理)により,RRに適合した形式で発給される許可書なしに,送信局を設置または運用してはならない」とされています(ただし例外として RR 18.2 / 18.8 / 18.11 がある旨も併記).講義資料側でも,RRの学習ポイントとして「電波を使用する条件として『局の許可書』が必要」と整理されています.

(設置の前提として)周波数の割当て・使用はRRに従う必要である.加盟国の主管庁は,原則としてRRの周波数分配表や他の規定に反する形で周波数を割り当ててはならず,例外的に反する割当てをする場合でも「有害な混信を与えない/与えられても保護を要求しない」という条件付きである,とされています.

この論点が “送信局の設置=許可が要る” だと講義でも明示されており,「RR18.1により送信局の設置または運用には『国の許可書』が必要」と設問でも明記されています.

\subsubsection*{引用}
\begin{itemize}
  \item 資料3.pdf p.30(RR 18.1 “Licenses”,および RR 4.4 “Assignment and use of frequencies”)
  \item レジメ2.pdf p.2(学習ポイント:「局の許可書」が必要)
  \item 資料4.pdf p.41(設問中の明示:「RR18.1により送信局の設置または運用には国の許可書が必要」)
\end{itemize}

\subsection{電気通信の秘密の確保について,ITU の基本文書にはどのような規定がありますか?}
\begin{itemize}
  \item \textbf{(ITU憲章)国際通信の秘密の保護の確保}\\
        構成国は,国際通信の秘密の保護を確保するための措置をとることを約束しています(憲章 第37条)。

  \item \textbf{(無線通信規則 RR)無線通信の秘密(第V章 第17条:RR17.1--17.4)}\\
        主管庁(Administrations)は,条約の関連規定を適用するに当たり,次の行為を禁止し,防止するために必要な措置を執ることを約束します(RR17.1--17.3)。
        \begin{itemize}
          \item 公衆の一般的利用を目的としない無線通信を,許可なく傍受すること.
          \item 傍受によって得られた情報について,許可なく,その内容または単にその存在を漏らすこと,公表すること,または利用すること.
        \end{itemize}
        また,局の許可書を有する者は,憲章・条約の関連規定に従って電気通信の秘密を守ることを要するとされます.
        さらに,傍受する意思がなく偶然に上記に該当する無線通信を受信した場合でも,再生して第三者に通知したり,いかなる目的にも使用したりしてはならず,その存在も漏らしてはならない(RR17.4)とされています.
\end{itemize}

\subsubsection*{引用}
\begin{itemize}
  \item 資料2.pdf(通信法規 p.11)「電気通信の秘密」:憲章 第37条「国際通信の秘密の保護を確保するための措置をとる」部分.
  \item 資料2.pdf(通信法規 p.21 付近)「無線通信の秘密の確保(第V章 第17条)」:RR17.1--17.3(無断傍受の禁止/内容・存在の漏えい・公表・利用の禁止),および許可書保有者の秘密保持.
  \item 資料2.pdf(通信法規 p.21)RR17.4(偶然受信でも,再生して第三者通知・目的使用・存在漏えいを禁止)。
\end{itemize}

\section{チェックポイント2}
\subsection{電波法に定める次の用語の定義}
電波法第2条(定義)により,次のとおりである.
\begin{enumerate}
  \item 電波:三百万メガヘルツ以下の周波数の電磁波.
  \item 無線設備:無線電信,無線電話その他電波を送り,または受けるための電気的設備.
  \item 無線局:無線設備及び無線設備の操作を行う者の総体.ただし,受信のみを目的とするものを含まない.
\end{enumerate}

\subsubsection*{引用}
\begin{itemize}
  \item 資料3.pdf(通信法規 p.22)「2 定義/1)電波法第2条(定義)」の箇所(電波,無線設備,無線局の定義):
  \item 資料5.pdf(通信法規 p.42)「無線設備:電波を送り,または受けるための電気的設備(電波法第2条)」:
\end{itemize}

\subsection{「無線局の開設」とはどのようなことを言いますか?}
「無線局の開設」とは,\textbf{運用する意思を持って,無線設備を設置し,その設備の操作をする者を配置して,電波の発射が可能な状態にすること}をいいます.

また,無線局の開設には原則として\textbf{免許・登録など一定の手続き(要式行為)}が必要であり,制度として\textbf{免許/許可制}(一定の事項を法的に禁止しておき,条件充足で禁止を解除)と\textbf{登録制}(公簿への記載により法律上の効力を発生)が説明されています.

\subsubsection*{引用}
\begin{itemize}
  \item 資料3.pdf(通信法規 p.22--23)「無線局の開設」:開設の定義(運用意思・無線設備の設置・操作者の配置・電波発射可能状態)。
  \item 資料3.pdf(通信法規 p.22--23)「免許/許可制」「登録制」の説明.
  \item レジメ3.pdf(通信法規\_03 p.5)「無線局の開設(第4条)」:免許/登録の区分(免許/許可制・登録制)。
  \item 資料3.pdf(通信法規 p.23)無線局の開設に必要な手続(免許・登録等の要式行為)に関する説明.
\end{itemize}

\subsection{「簡易な免許手続き」により開設できる無線局にはどのようなものがありますか?}
電波法第15条に基づく「簡易な免許手続き」により開設できる無線局は,次のものです.

\begin{itemize}
  \item 再免許(再免許の対象となる無線局)
  \item 特定無線局
  \item 遭難自動通報局
  \item 特定実験試験局
\end{itemize}

また,資料では「簡易な免許手続き」の対象を,よりまとめて
「(1) 適合表示無線設備のみを使用する無線局,(2) 遭難自動通報局等,(3) 特定実験試験局」
として示しています.

ここで「特定無線局」とは,適合表示無線設備のみを使用する無線局であって,例えば次のいずれかに該当するものです(施行規則第15条の2)。
\begin{itemize}
  \item 電気通信業務を行うことを目的とする陸上移動局,VSAT地球局,航空機地球局,携帯移動地球局
  \item MCA/デジタルMCA陸上移動通信を行う陸上移動局
  \item 実数零点単側波帯変調方式・狭帯域デジタル通信方式の陸上移動局および携帯局
  \item 屋内その他,混信その他の妨害を与えるおそれがない場所に設置する基地局
\end{itemize}

\subsection{同一機種かつ同一ロットで製造された携帯電話機(性能・規格が同一の)10万台について 無線局の開設手続きをする場合,どのような手続きによるのが適切ですか?}
同一機種かつ同一ロットで製造され,性能・規格が同一の携帯電話機を10万台まとめて無線局として開設手続きする場合は,\textbf{「包括免許の手続き(特定無線局の免許の特例)」}によるのが適切です.特定無線局であって,\textbf{目的・通信の相手方・電波の型式及び周波数・無線設備の規格が同一のものを二以上開設}する場合に,包括免許を申請でき,例として\textbf{携帯電話機}が挙げられています.

\subsubsection*{引用}
\begin{itemize}
  \item レジメ4.pdf(通信法規\_04)13頁:包括免許の免許手続(第27条の2)として,特定無線局で「目的,通信の相手方,電波の型式及び周波数並びに無線設備の規格が同一のものを2以上開設」する場合に包括免許を申請できる旨
  \item レジメ3.pdf(通信法規\_03)8頁:包括免許の手続き(電波法第27条の2から第27条の6)とし,例として「携帯電話機,MCA移動端末など」を明記
\end{itemize}

\subsection{「空中線電力が1ワット以下のPHS の基地局」を開設する場合,どのような開設手続きをするのが適切ですか?}
「空中線電力が1ワット以下のPHSの基地局」は,「登録局の対象となる無線局」として挙げられているため,\textbf{免許ではなく「登録」}により開設するのが適切です(電波法第4条第1項第四号・電波法第27条の18)。

具体的には,\textbf{登録を要する無線局}とは,(i) 一定時間自己の電波を発射しないことを確保する機能等を有し,他の無線局の運用を阻害する混信等を与えないように運用できる無線局であって,(ii) \textbf{適合表示無線設備のみを使用}し,(iii) \textbf{総務省令で定める区域内}に開設しようとする者は,\textbf{総務大臣の登録を受けなければならない},と整理されています.

したがって,「空中線電力が1ワット以下のPHSの基地局」(無線設備規則第49条の8の3)は,\textbf{登録手続}(登録申請)で開設します.

登録手続きの流れは次のとおりです.
\begin{itemize}
  \item 申請:電波法第27条の18に基づき,申請書および添付書類を提出
  \item 受理:無線局免許手続規則第25条の12に基づく,要式(形式)審査
  \item 審査:電波法第27条の20に基づく,拒否事由該当性のチェック
  \item 登録の実施:電波法第27条の19に基づき,総合無線局管理ファイルに登録
\end{itemize}

\subsubsection*{引用}
\begin{itemize}
  \item 資料3(通信法規 p.25):登録を要する無線局の定義(電波法第4条第1項第四号・第27条の18,適合表示無線設備のみ,省令で定める区域内なら登録が必要)
  \item 資料3(通信法規 p.25):登録局の対象となる無線局の表(「空中線電力が1ワット以下のPHSの基地局」=無線設備規則第49条の8の3)
  \item 資料3(通信法規 p.26):登録の手続き(申請$\rightarrow$受理$\rightarrow$審査$\rightarrow$登録の実施)
\end{itemize}

\subsection{無線通信規則(RR18.1)の規定により送信局の設置または運用には「国の許可書」が必要です.
  ところが,条約優先の規定(電波法第3条)を持つ電波法において,我が国の許可書に該当する免許状または登録状のいずれも必要としない無線局の開設が認められています.
  どのような規定によりこのようなことが可能となるか,述べなさい.}

RR18.1は,送信局の設置・運用には「政府(または政府を代表して)発給する許可書(licence)」を要する旨を定めていますが,ここで要求されているのは「各国政府が,RRに適合するかたちで与える許可」であって,必ずしも\textbf{無線局ごとに免許状(個別免許状)を発給すること}や,\textbf{登録状による制度}に限定されません(許可の「様式」は各国の制度設計に委ねられる)。

したがって,日本の電波法が,所要の技術的・運用的条件の下で「免許及び登録を要しない無線局」(電波法第4条各号)を法定しても,それはRR18.1の「国が与える許可」を,\textbf{個別免許状の交付ではなく,一般的な制度(一般許可・包括的な許容)として実現している}ものと整理でき,条約優先(電波法第3条)と整合します.

加えて,RR側にも,個別免許状がない形態の送信局(例:無線LAN端末やRFID等)を説明するために考慮すべき規定として,次の枠組みが用意されています.
\begin{itemize}
  \item \textbf{RR4.4(周波数割当ての例外的取扱い)}\\
        周波数分配表やRRの他規定に反して周波数を割り当てることは原則できないが,例外として,その割当てを使用する局が\textbf{有害な混信を生じさせないこと},および\textbf{有害な混信からの保護を要求しないこと}を明示の条件とする場合は認める,という考え方が示されています.これは,免許不要局のように「保護を主張せず,混信を与えない範囲で使う」制度設計と整合します.
  \item \textbf{RR1.15 および RR5.138/5.150(ISM周波数帯の位置づけ)}\\
        電気通信以外の目的で局所的に高周波エネルギーを利用するISMの定義や,ISMに指定された周波数帯(例:2.4\,GHz帯を含む)が示され,またISM利用は\textbf{主管庁の特別の許可(special authorization)}に服すること等が定められています.免許不要局のうち,実質的にISM的利用やそれに近い性格をもつものについては,この枠組みと整合させて国内制度(技術基準適合・低出力・混信抑制等)として許容し得ます.
  \item \textbf{RR第III章(RR8.4,RR11.36)と国際登録の考え方}\\
        周波数割当ては,国際周波数登録原簿(MIFR)に記録されることで国際的に認知され,RRに規定する権利(保護の主張等)を主張できる,という整理が示されています.裏返せば,免許不要局のように国際的権利主張を予定しない運用(保護要求をしない・局所利用・低出力等)については,国際登録・通告の枠組みとの関係も踏まえつつ,国内的に「免許・登録を要しない」類型を置くことが制度的に説明できます.
\end{itemize}

以上より,日本で免許状・登録状を要しない無線局の開設が可能となる根拠は,
\begin{enumerate}
  \item RR18.1の「適当な様式の許可(licence)」が\textbf{個別免許状に限定されず},国内法で一般的に許容する方式(免許不要局制度等)でも満たし得ること,
  \item その具体的な制度設計が,RR4.4の「有害混信を与えない・保護を要求しない」という国際的整理や,RR1.15・RR5.138/5.150のISMの考え方,さらにRR第III章の国際登録(MIFR)と権利主張の考え方と整合するように構成できること,
  \item そして国内法上は,電波法第4条により「免許及び登録を要しない無線局」を明文で類型化していること,
\end{enumerate}
にある,と述べられます.

\subsubsection*{引用}
\begin{itemize}
  \item 資料2.pdf 20頁:周波数割当てがMIFRに記録されることで国際的に認知され,RR上の権利主張が可能となる旨(「周波数割当て及び計画変更の手続き」)。
  \item 資料2.pdf 21頁:「局の許可書」(RR18.1)の要旨,および免許状のない送信局の存在と,考慮すべき規定としてRR1.15, RR4.4, RR5.150/5.138, RR8.4, RR11.36が挙げられている箇所(併せてRR4.4の和文要旨)。
  \item 資料3.pdf 30頁:RR18.1(licence in an appropriate form),RR4.4(有害混信を与えない・保護を要求しない条件),RR1.15(ISM定義),RR5.138/5.150(ISM指定周波数帯とspecial authorization)の英文抜粋.
  \item レジメ3.pdf 12--14頁:電波法第4条に基づく「免許及び登録を要しない無線局」(微弱無線局,市民ラジオ等)の整理.
  \item レジメ5.pdf 15頁:免許不要局(例:2.4\,GHz帯小電力データ通信システム等)を含む特定無線設備の例示.
\end{itemize}

\subsection{電波法で定める「無線局の不法な開設」に当たるケースにはどのようなものがありますか?}
電波法上の「無線局の不法開設」に当たる代表例は,次のとおりです.
\begin{itemize}
  \item \textbf{免許が必要なのに免許を受けずに開設した場合}\\
        (電波法第4条により免許を要する無線局を,免許なく開設)

  \item \textbf{登録が必要なのに登録を受けずに開設した場合}\\
        (電波法第27条の18第1項により登録を要する無線局を,登録なく開設)

  \item \textbf{包括免許の範囲(指定無線局数)を超えて開設した場合}\\
        (包括免許の免許記録に記載された指定無線局数を超えて,当該包括免許に係る特定無線局を開設)
\end{itemize}

また,不法開設は\textbf{実際に無線設備の運用(電波発射)をしたか否かにかかわらず処罰対象}となり,不法に無線局を開設した者は\textbf{一年以下の懲役または100万円以下の罰金}とされます.

さらに,不法無線局対策の文脈では,\textbf{電波法第4条違反で開設された無線局のうち,特定の周波数帯(26.1MHz超28MHz未満,144MHz以上146MHz以下,430MHz以上440MHz以下,889MHz超911MHz未満)を使用するもの}が「\textbf{特定不法開設局}」として整理されています.

\subsubsection*{引用}
\begin{itemize}
  \item 資料3.pdf(通信法規 p.31--32)「2 無線局の不法開設(電波法第110条第一号,第二号,第三号関連)」の箇所
  \item レジメ8.pdf(p.8)「3 不法無線局対策」中の「特定不法開設局(周波数帯の列挙)」の箇所
\end{itemize}

\subsection{無線局の免許の絶対的欠格事由とは何か?}
無線局免許の\textbf{絶対的欠格事由}とは,\textbf{一定の範囲の外国性を排除するため}に設けられている欠格事由であり,これに該当する者には\textbf{無線局の免許を与えてはならない(免許を受けることができない)}とされるものです(電波法第5条第1項).具体的には,次の者が該当します.
\begin{itemize}
  \item 日本の国籍を有しない人
  \item 外国政府またはその代表者
  \item 外国の法人または団体
  \item 法人または団体であって,前三号に掲げる者がその代表者であるもの,またはこれらの者がその役員の3分の1以上もしくは議決権の3分の1以上を占めるもの
\end{itemize}

なお,\textbf{外国性の排除(上記の絶対的欠格事由)の適用が例外的に除外される無線局}が定められており(電波法第5条第2項),例えば実験無線局,アマチュア無線局,船舶・航空機の航行安全のための無線局,一定の固定地点間通信の無線局,大使館等の公用で相互主義によるもの,自動車等に開設・携帯して使用する無線局やそれらと通信するための陸上局(一定のもの),電気通信業務目的の無線局,電気通信業務用無線局を搭載する人工衛星の制御用地上局等については,この外国性欠格が適用されません.

\subsubsection*{引用}
\begin{itemize}
  \item レジメ3.pdf 16ページ:無線局免許の絶対的欠格事由(一定の範囲の外国性排除)と該当者の列挙
  \item 資料3.pdf 10ページ:3.1 無線局免許の絶対的欠格事由(電波法第5条第1項)の列挙
  \item レジメ3.pdf 17ページ:外国性の排除の例外(電波法第5条第2項)の対象無線局の列挙
  \item 資料3.pdf 11ページ:3.2 外国性の排除の例外(電波法第5条第2項)の対象無線局の列挙
\end{itemize}

\subsection{無線局の免許の相対的欠格事由とは何か?}
無線局免許における「相対的欠格事由」とは,電波利用の場の秩序維持(反社会性の排除)の観点から,\textbf{一定の条件に該当する者について,総務大臣が「免許を与えないことができる」}とされる欠格事由をいう(絶対的欠格事由のように「免許を与えない(与えてはならない)」と機械的に排除するものではなく,裁量的に免許を与えないことができる類型である)。

具体的には(電波法第5条第3項),次の者が相対的欠格事由に該当する.
\begin{itemize}
  \item 電波法または放送法に規定する罪を犯し,罰金以上の刑に処せられ,その執行を終わり,またはその執行を受けることがなくなった日から2年を経過しない者
  \item 無線局の免許の取消しを受け,その取消しの日から2年を経過しない者
  \item 認定の取消しを受け,その取消しの日から2年を経過しない者
  \item 登録の取消しを受け,その取消しの日から2年を経過しない者
\end{itemize}

\subsubsection*{引用}
\begin{itemize}
  \item 資料3.pdf 10ページ:欠格事由の位置付け(相対的=「免許を与えないことができる」)の整理
  \item 資料3.pdf 11ページ:無線局免許の相対的欠格事由(電波法第5条第3項)の具体的類型(4項目)
  \item レジメ3.pdf 18ページ:無線局免許の相対的欠格事由(反社会性の排除:第5条第3項)の要点整理(4項目)
\end{itemize}

\subsection{一般の無線局と放送局の免許の欠格事由の違いは何か?}
一般の無線局(放送局を除く.)の免許欠格事由は,電波法第5条により,
\begin{itemize}
  \item \textbf{絶対的欠格事由(一定の範囲の外国性の排除)}:該当すると免許を与えない(例:外国人,外国政府,外国法人等,およびそれらが代表者・役員の3分の1以上または議決権の3分の1以上を占める法人等)
  \item \textbf{相対的欠格事由(反社会性の排除)}:該当すると免許を与えないことができる(例:電波法・放送法違反で罰金以上,免許取消等から2年未経過 など)
  \item \textbf{外国性排除の例外}:一定の無線局(実験等無線局,アマチュア局等)には外国性排除の欠格事由が適用されない
\end{itemize}
という構造で整理されます.

これに対し\textbf{放送局}は,電波法第5条第4・5項により\textbf{「欠格事由の厳格化」}があり,一般の無線局より欠格の範囲が広く(厳しく)定められます.具体的には,
\begin{itemize}
  \item \textbf{外国性排除(第5条第1項第1号から第3号)に該当する者}に加えて,\textbf{一般の無線局では「相対的欠格事由」に該当する者}も放送局では欠格として扱われる.
  \item \textbf{法人・団体についての基準が厳しい}:外国人等(第5条第1項第1号から第3号)について,\textbf{業務執行役員である場合}や,\textbf{議決権の1/5以上を占める場合}などが欠格となる(一般の無線局の外国性基準が「3分の1以上」であるのに対し,放送局は「1/5以上」とされている点が大きな相違)。
  \item \textbf{役員に相対的欠格事由該当者がいる法人・団体}も欠格となる.
\end{itemize}

要するに,\textbf{一般の無線局は「外国性(絶対)」+「反社会性(相対)」}を基本とするのに対し,\textbf{放送局は相対的欠格事由の対象も取り込みつつ,法人に対する外国関与の判定も「3分の1」から「1/5」へと厳格化するなど,欠格事由が拡張・強化されている}点が違いです.

\subsubsection*{引用}
\begin{itemize}
  \item レジメ3:16ページ(欠格事由の趣旨,無線局免許の絶対的欠格事由(外国性排除:役員・議決権3分の1以上))
  \item レジメ3:17ページ(外国性排除の例外(第5条第2項))
  \item レジメ3:18ページ(無線局免許の相対的欠格事由,放送局の免許の欠格事由(第5条第4・5項:厳格化,議決権1/5以上等))
  \item 資料3:32ページ(欠格事由の目的整理(外国性=絶対/反社会性=相対),外国性排除(役員・議決権3分の1以上))
\end{itemize}

\subsection{無線局予備免許の指定事項を列挙しなさい.}
無線局の予備免許で指定される事項は,次のとおりである.
\begin{itemize}
  \item 工事落成の期限
  \item 電波の型式及び周波数
  \item 識別信号
  \item 空中線電力
  \item 運用許容時間
\end{itemize}

\subsubsection*{引用}
\begin{itemize}
  \item 資料4.pdf p.35「4.4 予備免許(電波法第8条)/指定事項」
  \item レジメ4.pdf p.7「予備免許(第8条)/次の事項を指定」
\end{itemize}

\subsection{無線局の免許手続きでいう「工事設計の変更」とはどのようなことですか?}
無線局の免許手続(通常の免許手続)では,申請・審査を経て予備免許が付与された後,落成届・落成検査に進むまでの間に,申請時に提出した「工事設計(工事設計書の内容)」等を変更したくなる場合があります.これが免許手続上の「工事設計の変更」です.

この「工事設計の変更」は,手続上つぎの3つに整理されます.
\begin{itemize}
  \item \textbf{総務大臣の事前の許可を要する変更}\\
        通信の相手方,通信事項(放送事項),放送区域,無線設備の設置場所などに関する変更は,事前に総務大臣の許可が必要です.
  \item \textbf{総務大臣の事前の許可を要しない変更(軽微なこと項の変更)}\\
        電波法施行規則第10条および同規則別表第一号の三に掲げる事項の変更は「軽微なこと項の変更」とされ,事前許可は不要ですが,\textbf{遅滞なく届出}を行います.
  \item \textbf{変更できない事項}\\
        周波数・電波の型式・空中線電力に変更を来すもの,または電波法第三章の技術基準に合致しないものとなる変更は,認められません.
\end{itemize}

\subsubsection*{引用}
\begin{itemize}
  \item 資料4.pdf p.35「4.5 工事設計の変更(電波法第9条,電波法施行規則第10条,無線局免許手続規則第12条)」の整理(事前許可が必要な変更/軽微変更(届出)/変更できない事項)
  \item レジメ4.pdf p.2「通常の免許手続き」の手続フロー(予備免許の後に「工事設計の変更(第9条)」が位置づく)
\end{itemize}

\subsection{無線局の免許手続きにおいて免許が拒否されるのはどのような場合ですか?}
無線局の免許手続においては,総務大臣は次の場合に無線局の免許を拒否しなければなりません.
\begin{itemize}
  \item 免許申請を審査した結果,\textbf{予備免許の付与の要件に適合しない}場合.
  \item \textbf{工事落成期限経過後2週間以内に工事落成届が提出されない}場合.
  \item 落成後の検査の結果,\textbf{不合格となった}場合.
\end{itemize}

\subsubsection*{引用}
\begin{itemize}
  \item 資料4.pdf(通信法規 p.36)「無線局の免許の拒否の理由」:免許拒否となる3つの場合の列挙.
  \item 資料4.pdf(通信法規 p.35--36)「4.7 免許の拒否(電波法第11条,無線局免許手続規則第14条)」:総務大臣が免許を拒否しなければならない旨.
\end{itemize}

\subsection{無線局免許状の記載事項を列挙しなさい.}
無線局免許状(免許記録)に記載される事項は,次のとおりである.
\begin{itemize}
  \item 免許の年月日及び免許番号
  \item 免許人の氏名または名称及び住所
  \item 無線局の種別
  \item 無線局の目的
  \item 通信の相手方及び通信事項(放送局の場合は放送事項及び放送区域)
  \item 無線設備の設置場所
  \item 免許の有効期間
  \item 識別信号
  \item 電波の型式及び周波数
  \item 空中線電力
  \item 運用許容時間
\end{itemize}

\subsubsection*{引用}
\begin{itemize}
  \item レジメ4.pdf(通信法規\_04のスライド12):免許記録(免許状)の記載事項の列挙
  \item 資料4.pdf(通信法規 p.36):免許状記載事項の列挙
\end{itemize}

\subsection{特に規定の無い場合,無線局の無線局免許状の有効期限は何年ですか?}
特に規定の無い無線局の免許状の有効期限は,\textbf{5年}です.
これは,免許の有効期間が「免許の日から起算して五年を超えない範囲内において」定められる(電波法第13条第1項)ことを前提に,種別ごとの整理として「特に規定の無い無線局:5年」と示されているためです.

\subsubsection*{引用}
\begin{itemize}
  \item 資料4.pdf(通信法規 p.36)「4.9 免許の有効期限」:免許の有効期間は原則5年以内,および表「特に規定の無い無線局 5年」
  \item レジメ4.pdf(通信法規\_04 p.11)表「有効期限/無線局の種別」:特に規定の無い無線局は5年
\end{itemize}

\subsection{特に規定の無い場合,無線局の再免許の申請時期は何時ですか?}
再免許申請は,免許の有効期間満了前に行わなければならず,\textbf{特に規定の無い無線局}については,\textbf{満了前3箇月以上6箇月を超えない期間}に申請します.あわせて,再免許には\textbf{免許内容の変更を含めることは認められません}。

(参考:無線局の種別ごとの申請時期(いずれも免許の有効期間満了前))
\begin{itemize}
  \item 特に規定の無い無線局:3箇月以上6箇月を超えない期間
  \item アマチュア局(人工衛星等のアマチュア局を除く):1箇月以上1年を超えない期間
  \item 特定実験局:1箇月以上3箇月を超えない期間
  \item 免許の有効期間が1年以内である無線局:1箇月までに
  \item (免許終期の統一に関連し)満了前1箇月以内に免許を与えられた無線局:免許を受けた後直ちに
\end{itemize}

\subsubsection*{引用}
\begin{itemize}
  \item レジメ4.pdf(通信法規\_04)p.15「再免許」:申請時期(特に規定の無い無線局は満了前3箇月以上6箇月を超えない期間),および「免許の内容の変更を含む再免許は認められない」
  \item 資料4.pdf(通信法規)p.37「再免許」:再免許申請の申請時期(特に規定の無い無線局は満了前3か月以上6か月を超えない期間)等の一覧
\end{itemize}

\subsection{免許人の地位の承継とはどのようなことをいいますか?}
免許人の地位の承継とは,\textbf{ある者が,他の者の法令上の権利及び義務を受け継ぐこと(承継)}であり,無線局免許に関しては,その\textbf{免許人としての地位(権利義務のセット)}を一定の原因により他者が引き継ぐことをいう.

承継が生じる典型例(承継の原因)と,承継する者・手続は次のとおりである.
\begin{itemize}
  \item \textbf{相続(免許人についての相続)}\\
        承継する者:相続人.\\
        手続:遅滞なく,その事実を証する書面を添えて総務大臣に\textbf{届出}。
  \item \textbf{法人免許人の合併または分割}\\
        承継する者:合併後存続する法人/合併により設立された法人/分割により当該事業の全部を承継した法人のうち,\textbf{総務大臣の許可を受けたもの}。\\
        手続:\textbf{許可の申請}。
  \item \textbf{無線局を使用して行う事業の譲渡}\\
        承継する者:譲受人のうち,\textbf{総務大臣の許可を受けたもの}。\\
        手続:\textbf{許可の申請}。
  \item \textbf{船舶及び航空機の運航者の変更}\\
        承継する者:変更後に船舶または航空機を運航する者.\\
        手続:遅滞なく,その事実を証する書面を添えて総務大臣に\textbf{届出}。
\end{itemize}

また,\textbf{法人の合併または分割}および\textbf{事業の譲渡}による承継の\textbf{許可}については,\textbf{免許の欠格事由(電波法第5条)}および\textbf{免許申請の審査(電波法第7条)}の規定が準用される.

\subsubsection*{引用}
\begin{itemize}
  \item 資料4.pdf(通信法規 p.38)「6 免許人の地位の承継(電波法第20条,無線局免許手続規則第二章第二節の二)」の定義,承継原因ごとの承継者・手続(届出/許可申請),準用規定,届出義務(遅滞なく・証明書面添付)
  \item レジメ4.pdf(通信法規\_04 p.15)「免許の承継」スライドの整理(承継の定義,原因別の承継者と手続)
\end{itemize}

\subsection{無線局免許が失効する場合の原因を述べなさい.}
(包括免許の場合を除く)無線局免許が失効する原因は,次のとおりである.
\begin{itemize}
  \item 有効期間の満了(電波法第13条)
  \item 無線局の廃止(電波法第23条)
  \item 外国で取得した船舶局または航空機局について特例により免許が与えられている場合で,当該船舶または航空機が日本国内の目的地に到着したとき(電波法第27条第2項)
  \item 免許の取消し(電波法第75条,第76条)
  \item 免許人が存在しなくなったとき(法人の解散,免許人が死亡し免許人の地位の承継が行われない場合)
  \item 多重放送の無線局で,本体放送局の免許が効力を失ったとき(電波法第13条の2)
\end{itemize}

\subsubsection*{引用}
\begin{itemize}
  \item レジメ4.pdf(通信法規\_04)p.15--16「免許の失効及び取消し/無線局免許の失効の原因(包括免許の場合を除く)」
  \item 資料4.pdf(通信法規)p.38「無線局の免許の失効及び取消し(包括免許の場合を除く)/無線局免許の失効の原因」
\end{itemize}

\section{チェックポイント3}
\subsection{「電波の型式」とは何ですか?}
電波法施行規則における「電波の型式」とは,その電波がどのような情報を,どのような変調で搬送波に載せて伝送するかを表すための表示であり,
\begin{itemize}
  \item 「主搬送波の変調の型式」
  \item 「主搬送波を変調する信号の性質」
  \item 「伝送情報の型式」
\end{itemize}
の3要素を組み合わせた(三桁の英数字からなる)記号として表されます.さらに無線通信規則(RR)では,上の3要素を「基本的特性」とし,これに「付加特性(信号の詳細+多重の種類)」を加えたものを「発射の種別」として扱います.

また,特定の電波を表示する際は,原則として「電波の型式」+『電波』+「周波数」の順序で表し(例:A3E 電波 1000kHz),無線局免許申請書等でも「電波の型式並びに希望する周波数の範囲及び空中線電力」が記載事項として求められます.

\subsubsection*{引用}
\begin{itemize}
  \item 資料5.pdf p.42:電波の表示の通則/電波の型式の3要素,RRにおける基本的特性・付加特性と「発射の種別」
  \item レジメ5.pdf p.5:電波の表示の順序(「電波の型式」+『電波』+「周波数」),電波の型式の定義(三桁の英数字),3要素の構成
  \item レジメ4.pdf p.4:無線局免許申請書の記載事項として「電波の型式並びに希望する周波数の範囲及び空中線電力」
\end{itemize}

\subsection{「空中線電力」と「輻射電力」の違いを述べなさい.}
空中線電力は,\textbf{送信機から送信空中線の給電線に供給される電力}であり,送信装置(送信機)側から空中線系へ「投入される電力」を表す.

これに対して輻射電力は,\textbf{空中線に供給される電力}に\textbf{空中線利得(与えられた方向の利得)}を掛け合わせた量として定義され,空間へ「どれだけ強く放射されるか(特定方向での放射の強さ)」を表す.具体的には,
\begin{itemize}
  \item 等価等方輻射電力(e.i.r.p.):空中線に供給される電力 $\times$ 与えられた方向における\textbf{絶対利得}
  \item 実効輻射電力(e.r.p.):空中線に供給される電力 $\times$ 与えられた方向における\textbf{相対利得}
\end{itemize}
である.

したがって両者の違いは,\textbf{空中線電力が「給電線に投入される電力(入力側)」の量であるのに対し,輻射電力は「投入電力に空中線利得を反映した放射側(特定方向)の量」}である点にある.

\subsubsection*{引用}
\begin{itemize}
  \item 資料5.pdf(通信法規\_05),p.47:空中線電力の定義(「送信機から送信空中線の給電線に供給される電力」)
  \item 資料5.pdf(通信法規\_05),p.47:輻射電力の定義(e.i.r.p./e.r.p.:空中線に供給される電力と利得の積)
  \item レジメ5.pdf(通信法規\_05),p.10:空中線電力と輻射電力(e.i.r.p., e.r.p., e.m.r.p.)の整理
\end{itemize}

\subsection{電波法に定める「電波の質」について述べなさい.}

電波法は,送信設備から発射される電波について,他の無線局への妨害を防止し,円滑な通信の確保および良好な電波利用環境を維持する観点から,「電波の質」を一定の基準に適合させることを求めている.すなわち,周波数帯の有効利用(必要以上の周波数領域を占有しない)と,電波利用環境の保護(不要な成分の少ない「きれいな電波」を出す)を目的として規定される.

電波法第28条(電波の質)により,送信設備に使用する電波の
「周波数の偏差及び幅」ならびに「高調波の強度等(不要発射等)」といった電波の質は,総務省令で定めるところに適合しなければならない.

電波の質として具体的に管理される主なこと項は次のとおりである.
\begin{itemize}
  \item 周波数の偏差(割当周波数等からのずれ)
  \item 占有周波数帯域幅(その発射が占有する周波数の幅)
  \item 不要発射の強度(例:帯域外発射,スプリアス発射)
  \item スプリアス発射の内訳としての高調波発射・低調波発射・寄生発射(寄生振動)・相互変調積
  \item 隣接チャネル漏えい電力(システムごとに規定)
\end{itemize}

また,無線局の発射する電波の質が無線設備規則の基準に適合していないと認められる場合には,電波法第72条により,総務大臣は当該無線局に対して電波の発射停止を命ずることができる(適合した場合は停止命令を解除する)。

\subsubsection*{引用}
\begin{itemize}
  \item レジメ5.pdf(通信法規\_05)p.7--8:電波の質を規定する必要性,電波法第28条の要旨,電波の質の構成(周波数偏差・占有帯域幅・不要発射等)
  \item 資料5.pdf p.44:電波法第28条(電波の質)の条文要旨,電波の質として管理する項目(周波数偏差・占有帯域幅・スプリアス発射強度等)
  \item 資料5.pdf p.46:不要発射(帯域外発射・スプリアス発射等)の説明,必要周波数帯幅の説明
  \item レジメ7.pdf(通信法規\_07)p.4:電波法第72条(電波の発射停止)と「電波の質」不適合時の措置
\end{itemize}

\subsection{受信設備の条件及び性能としてどのようなことが求められているか述べなさい.}
受信設備については,大きく
\begin{itemize}
  \item 他の無線設備へ妨害を与えないための「条件」
  \item 受信を良好に行うための「性能」
\end{itemize}
が求められる.

\begin{itemize}
  \item \textbf{(条件)副次的に発する電波または高周波電流の限度}\\
        受信設備は,その副次的に発する電波または高周波電流が,総務省令で定める限度を超えて他の無線設備の機能に支障を与えるものであってはならない.\\
        具体的には,受信空中線と電気的定数の等しい疑似空中線回路で測定した場合に,その回路の電力が \textbf{四ナノワット以下}であることが求められる.

  \item \textbf{(性能)受信設備として備えるべき基本的性能}\\
        受信設備は,なるべく次の各事項に適合することが求められる.
        \begin{enumerate}
          \item 内部雑音が小さいこと
          \item 感度が十分であること
          \item 選択度が適正であること
          \item 了解度が十分であること
        \end{enumerate}

  \item \textbf{(受信空中線)受信空中線の扱い}\\
        受信空中線については,\textbf{送信空中線系の条件が準用}される.
\end{itemize}

\subsubsection*{引用}
\begin{itemize}
  \item 資料5.pdf:p.49「4 受信設備の条件(電波法第29条,無線設備規則第24条)」「4.2 受信設備の性能(無線設備規則第25条)」「4.3 受信空中線(無線設備規則第26条)」
\end{itemize}

\subsection{無線設備の安全施設について電波法令にはどのような規定がありますか?}
電波法令(電波法施行規則)では,無線設備について,その設備の操作等に伴って\textbf{人体に危害を及ぼし,または物件に損傷を与えることがないように}するため,\textbf{総務省令で定める安全施設}を設けることを求めています(施行規則第21条の2から第27条の施設)。

この「安全施設」の具体例として,電波法施行規則には少なくとも次のような規定があります.
\begin{itemize}
  \item \textbf{空中線等の保安施設(施行規則第26条)}\\
        無線設備の空中線系には\textbf{避雷器または接地装置}を設け,また\textbf{カウンターポイズには接地装置}を設けなければならない(ただし,\textbf{26.175\,MHzを超える周波数を使用する無線局の無線設備}および\textbf{陸上移動局または携帯局の無線設備の空中線系}については強制されない)。
  \item \textbf{航空機用気象レーダーの安全施設(施行規則第27条)}\\
        航空機用気象レーダーは,その設備の操作に伴って人体に危害を及ぼし,または物件に損傷を与えるおそれがある場合は,\textbf{必要と認められる施設}をしなければならない.
\end{itemize}

なお,同じく施行規則では\textbf{高周波利用設備}についても「安全施設」が整理されており(施行規則第47条から第50条),\textbf{許可を要する通信設備}には無線設備の安全施設(施行規則第2章第3節,すなわち第21条の2から第27条)の規定が準用されます.さらに,\textbf{医療設備}では高圧部の収容や医療電極まわりの絶縁等の条件,\textbf{工業用加熱設備・各種設備}では高圧部の収容に加えて,近接する人体・電気的良導体に高周波電力を誘発するおそれがあるときの危険防止設備などが条件として示されています.

\subsubsection*{引用}
\begin{itemize}
  \item 資料5.pdf 9ページ:5.4「空中線等の保安施設(電波法施行規則第26条)」/5.5「航空機用気象レーダーの安全施設(電波法施行規則第27条)」
  \item レジメ7.pdf 15ページ:通信設備の安全施設(施行規則第47条)として,無線設備の安全施設(施行規則第21条の2から第27条)を準用する旨
  \item レジメ7.pdf 14--16ページ:高周波利用設備の安全施設(施行規則第47条から第50条)として,医療設備・工業用加熱設備・各種設備の安全条件
\end{itemize}

\subsection{次の各制度について,対象となる無線設備,申請をする者,証明/認証/確認をする者,証明/認証/確認する事柄,を述べなさい.
  1)技術基準適合証明 2)工事設計認証 3)技術基準適合自己確認}

\begin{itemize}
  \item \textbf{1)技術基準適合証明}
        \begin{itemize}
          \item 対象となる無線設備:\textbf{特定無線設備}
          \item 申請をする者:\textbf{特定無線設備の製造・輸入・販売等の業者},および\textbf{特定無線設備を使用しようとする個人(一般利用者)}
          \item 証明をする者:\textbf{登録証明機関}(国外の場合は\textbf{承認証明機関})
          \item 証明する事柄:\textbf{個々の特定無線設備}を審査し,当該\textbf{特定無線設備が技術基準(電波法第三章)に適合}していることを\textbf{証明}する.
        \end{itemize}

  \item \textbf{2)工事設計認証}
        \begin{itemize}
          \item 対象となる無線設備:\textbf{特定無線設備}
          \item 申請をする者:\textbf{特定無線設備を取り扱うことを業とする者}
                (例:\textbf{製造・輸入・販売・修理・点検等を行う業者}。
                \textbf{個人として特定無線設備を使用する者は利用できない}。)
          \item 認証をする者:\textbf{登録証明機関}(国外の場合は\textbf{承認証明機関})
          \item 認証する事柄:電波法第三章の技術基準に適合する特定無線設備を生産するものとしての
                \textbf{工事設計(工事設計に合致することの確認方法を含む)}および\textbf{品質管理の方法}を審査し,\textbf{認証}する.
                (認証された工事設計・品質管理の方法=\textbf{認証工事設計},認証を受けた者=\textbf{認証取扱業者}。)
        \end{itemize}

  \item \textbf{3)技術基準適合自己確認}
        \begin{itemize}
          \item 対象となる無線設備:\textbf{特別特定無線設備}
                (技術基準・使用態様等を勘案し,\textbf{混信その他の妨害を与えるおそれが少ない}ものとして総務省令で定めるもの)
          \item 申請(届出)をする者:\textbf{製造業者または輸入業者}
          \item 確認をする者:\textbf{製造業者または輸入業者が自ら}確認し,\textbf{総務大臣に届出}る.
          \item 確認する事柄:当該\textbf{特別特定無線設備が技術基準(電波法第三章)に適合}していることを\textbf{自ら確認}し,
                その前提となる\textbf{工事設計および品質管理の方法}を含めて届出(届出に係る工事設計・品質管理の方法=\textbf{届出工事設計})。
                (届出をした者=\textbf{届出業者}。)
        \end{itemize}
\end{itemize}

\vspace{0.5em}
\noindent\textbf{(参考:表示との関係)}\\
上記の各制度により\textbf{証明・認証を受け,または自己確認の届出をした無線設備}には,
\textbf{技術基準適合マーク(技適マーク)}の貼付が許され,この表示の付された無線設備を\textbf{適合表示無線設備}という.

\subsubsection*{引用}
\begin{itemize}
  \item 資料5.pdf(通信法規 p.51):「技術基準認証制度」の概要(3制度の存在,対象(特定無線設備/特別特定無線設備),申請者,実施者,技適マーク・適合表示無線設備の位置づけ)
  \item 資料5.pdf(通信法規 p.51):「技術基準適合証明は個々の特定無線設備を審査し…証明」「個人も利用可」
  \item 資料5.pdf(通信法規 p.51):「技術基準適合証明及び工事設計認証の業務は登録証明機関(国外は承認証明機関)が行う」
  \item 資料5.pdf(通信法規 p.55):「工事設計認証」=工事設計・品質管理方法を審査し認証,利用できる申請者(個人は不可),認証工事設計・認証取扱業者
  \item 資料5.pdf(通信法規 p.55)電波法 第38条の24(工事設計認証の定義:登録証明機関が工事設計(確認方法含む)を認証)
  \item レジメ5.pdf(通信法規\_05 p.21--22)工事設計認証(工事設計・品質管理方法を認証,認証工事設計,認証取扱業者)/技術基準適合自己確認(総務大臣への届出,届出業者,届出工事設計)
  \item 資料5.pdf(通信法規 p.55)技術基準適合自己確認制度の説明(製造業者または輸入業者が自ら確認して総務大臣に届出,届出業者・届出工事設計)
  \item 資料5.pdf(通信法規 p.55)電波法 第38条の33(特別特定無線設備:混信等のおそれが少ないもの,製造業者または輸入業者が対象)
  \item 資料5.pdf(通信法規 p.51)技適マーク・適合表示無線設備の説明
\end{itemize}

\subsection{「適合表示無線設備」とはどのような無線設備ですか?}
適合表示無線設備とは,電波法の技術基準認証制度(技術基準適合証明・工事設計認証・技術基準適合自己確認)により,電波法第3章に定める技術基準に適合していることが確認され,その結果として「技術基準適合マーク(技適マーク)」の表示(貼付)が付された無線設備をいう.

また,MRA法(特定機器に係る適合性評価手続の結果の外国との相互承認の実施に関する法律)に基づく「登録外国適合性評価機関」が,技術基準に適合している旨の証明をした特定無線設備や,同機関が認証した工事設計に基づく特定無線設備も,「適合表示無線設備」とみなされる.

適合表示無線設備は,免許登録不要局・登録局・包括免許局の無線設備として位置づけられ,さらに「適合表示無線設備のみを使用する無線局」には簡易な免許手続が適用される.

\subsubsection*{引用}
\begin{itemize}
  \item 資料5.pdf 9ページ:技術基準適合マーク(技適マーク)が付された無線設備を「適合表示無線設備」ということ,およびMRA法に基づく登録外国適合性評価機関による証明・認証設備は「適合表示無線設備」とみなされること,並びに簡易な免許手続の適用に関する記述
  \item レジメ5.pdf 24ページ:登録外国適合性評価機関が証明した特定無線設備は適合表示無線設備とされること(MRAの説明を含む)
  \item レジメ5.pdf 16ページ:MRA法に規定する登録外国適合性評価機関が証明・認証した特定無線設備が「適合表示無線設備」とされること
\end{itemize}

\subsection{電波法で規定する「登録証明機関」とはどのような機関ですか?}
電波法の「登録証明機関」とは,\textbf{特定無線設備の技術基準適合証明(および工事設計認証に係る業務)を行うことについて,総務大臣の登録を受けた者(機関)}をいいます.
この登録証明機関が行う業務は,特定無線設備について,
\begin{itemize}
  \item 個々の設備が電波法第三章の技術基準に適合していることを\textbf{審査して証明する}(技術基準適合証明)
  \item 工事設計(設計・品質管理等)に基づく設備について\textbf{認証する}(工事設計認証)
\end{itemize}
といった,いわゆる\textbf{技術基準認証制度の実施主体}として位置づけられます.
また,\textbf{外国において同種の検査・試験等を行う者で,総務大臣の承認を受けた者}は「承認証明機関」とされ,国際的な相互承認(MRA)に基づき,登録外国適合性評価機関による証明・認証結果が,我が国の制度上「適合表示無線設備」とみなされる扱いもあります.
さらに,登録(承認)は,\textbf{対象となる特定無線設備の区分(事業の区分)}ごとに受ける形になっています.

\subsubsection*{引用}
\begin{itemize}
  \item 資料5.pdf p.51(技術基準適合証明・工事設計認証の実施者が登録証明機関であること,および適合表示無線設備との関係)
  \item 資料5.pdf p.56(登録証明機関の定義(法第38条の2の2),承認証明機関(法第38条の31),事業の区分と対象設備の例)
  \item レジメ5.pdf p.23(登録証明機関の定義(法第38条の2の2))
\end{itemize}

\section{チェックポイント4}
\subsection{無線設備の操作をする者は,原則として,どのような者でなければならないか?}
無線設備の操作は,原則として「無線従事者」が行うべきです.無線従事者とは,無線設備の操作またはその監督を行う者であって,総務大臣の免許を受けた者をいいます.

なお,例外として,主任無線従事者の監督の下で無資格者が(その資格の操作範囲内で)操作できる場合があります.また「簡易な操作」や「無線設備の操作の特例」の場合には,無資格者による操作が認められます.
一方で,モールス符号による無線電信通信のように一定の技能・知識を要する通信を行う場合,その無線設備の操作は無線従事者でなければ行えません.

\subsubsection*{引用}
\begin{itemize}
  \item 資料6.pdf:p.60「1.1 無線設備の操作の資格主義(電波法第39条)」に原則(無線従事者,または主任無線従事者の監督下での無資格者)と例外(簡易な操作・特例),および「無線従事者でなければ行ってはならない」場合の説明
  \item 資料3.pdf:p.22「無線従事者」の定義(無線設備の操作またはその監督を行う者であって,総務大臣の免許を受けたもの)
  \item レジメ6.pdf:p.7「無線従事者になるには,総務大臣の免許を受けなければならない」
\end{itemize}

\subsection{無線従事者の資格を得るには,どのような方法があるか?}
無線従事者になるには,総務大臣の免許を受ける必要があり,そのためにまず「無線設備の操作に必要な知識及び技能」の要件に適合していることを証明します.その知識・技能要件への適合の証明(=資格を得るためのルート)には,次の4つの方法があります.
\begin{itemize}
  \item 無線従事者国家試験に合格する.国家試験は資格別に,毎年少なくとも1回,総務大臣が行う.
  \item 養成課程を修了する.総務大臣の認定を受けた認定施設者が所定の授業を行い,修了試験に合格した者へ修了証明書等を交付する(養成課程).修了証明書等の交付を受けた者は,知識・技能要件に適合したものとされる.
  \item 学校教育法に基づく学校等を卒業する.大学・短大・高専・高校・中等教育学校等で所定の無線通信に関する科目を履修して卒業した者に,一定の無線従事者資格が付与される.
  \item 資格及び業務経歴を有したうえで認定講習課程を修了する.一定の無線従事者資格を所持し,その資格による実務経歴がある場合,所定の講習課程(認定講習課程)を修了することで上位の資格を取得できる.
\end{itemize}

なお,資格(知識・技能要件)を満たした後は,資格ごとに定められた申請書・添付書類を提出して無線従事者免許を申請し,欠格事由等に該当しなければ免許が与えられ,免許証の交付を受けます.

\subsubsection*{引用}
\begin{itemize}
  \item 資料6.pdf pp.65--66「無線従事者資格の取得」「知識及び技能の要件への適合の証明(4つの方法)」「国家試験・養成課程・学校卒業・認定講習課程の説明」
  \item レジメ6.pdf p.8「知識及び技能の要件への適合の証明としての4つの方法(国家試験・養成課程・学校卒業・認定講習課程)」
\end{itemize}

\subsection{「通信操作」及び「技術操作」について述べなさい.}
無線設備の操作は「通信操作」と「技術操作」に区分される.

\begin{itemize}
  \item \textbf{通信操作}:無線設備を作動させて,電鍵・送話器・自動送信機等を通じて行う送信行為,または受話器・拡声器等を通じて行う受信行為をいう.
  \item \textbf{技術操作}:無線設備を作動させ,またはその作動を停止させる行為(レーダ等による無線通信に関する行為を含む)に加え,無線設備の作動を最良にするための調整およびこれに付随する行為をいう.なお,無線設備の製造・建設・修理等は技術操作には含まれず,資格は必要としない.
\end{itemize}

\subsubsection*{引用}
\begin{itemize}
  \item 資料6.pdf(通信法規 p.60)「操作の種別(通信操作/技術操作)」および「製造・建設・修理は技術操作に含まれず資格不要」
  \item レジメ6.pdf(通信法規\_06 p.2)「操作の種別 ⇒ 通信操作と技術操作」の定義
\end{itemize}

\subsection{無線従事者の操作範囲とは何か?}
無線従事者の「操作範囲」とは,\textbf{その無線従事者資格で行うことが許される無線設備の操作(通信操作・技術操作)}の範囲をいい,\textbf{無線設備の操作の監督を行うことができる範囲も同一}とされます(政令で具体的に資格ごとに定められる).したがって,無線従事者(または主任無線従事者)が無線設備を操作・監督する場合は,\textbf{自分が所持する資格の操作範囲内}で行うのが原則です.

また,無線設備の操作は資格主義が採られており,原則として\textbf{無線従事者}が行うか,または\textbf{主任無線従事者の監督の下で無資格者が行う}ことになりますが,この場合でも\textbf{行えるのは無線従事者(または主任無線従事者)が所持する資格の操作範囲内}に限られます.

なお,無線従事者の資格自体が「操作または監督する無線設備」に対応して区分され(総合・海上・航空・陸上・アマチュアの区分など),その区分に応じて操作範囲(どの無線設備のどの操作までできるか)が定まります.

\subsubsection*{引用}
\begin{itemize}
  \item 資料6.pdf:PDF1ページ(資料表記60頁)「通信操作/技術操作の定義」「無線設備の操作は…資格の操作範囲内で行うのが原則」「電波法第39条(無線設備の操作)」周辺
  \item 資料6.pdf:PDF3ページ(資料表記62頁)「無線従事者の資格の区分(電波法第40条)…5区分,23資格」周辺
  \item 資料6.pdf:PDF4ページ(資料表記63頁)「無線従事者の操作及び監督の範囲(電波法施行令第3条)…操作できる範囲と監督できる範囲は同一」「操作及び監督の範囲(抜粋表)」周辺
\end{itemize}

\subsection{無線従事者の資格を持たなくても携帯電話を使うことができる理由はなにか?}
携帯電話を無線従事者の資格なしで使える理由は,電波法上「例外として,無線従事者でなくても行える操作」に携帯電話の利用形態が当てはまるからです.要点は次のとおりです.

\begin{itemize}
  \item 電波法では原則として,無線設備の操作は無線従事者が行う(または主任無線従事者の監督下で行う)という建付けですが,総務省令で定める「簡易な操作」などは例外として,資格も主任無線従事者の監督も不要とされています.
  \item その「簡易な操作」(電波法施行規則第33条)には,
        \begin{itemize}
          \item 「特定無線局」の無線設備の\textbf{通信操作}(および電波の質に影響を及ぼさない外部転換装置の技術操作)
        \end{itemize}
        が含まれます.携帯電話のユーザが通常行うのは,まさにこの範囲(通信の発着信等の通信操作)です.
  \item さらに,携帯電話機は「特定無線局」の典型例として扱われ,\textbf{適合表示無線設備(技適マーク付き)}のみを使用することが前提になっています.つまり,機器側が技術基準に適合していることが制度的に担保され,一般利用者が資格なしで操作しても無線の秩序・安全を損ねにくい仕組みになっています.
  \item 特定無線局は多数台をまとめて免許(包括して対象とする免許)を受けられる枠組みがあり,携帯電話機のように同一仕様の端末を多数運用する実態に合っています.そのため「個々の利用者が無線従事者資格を持って個別に操作する」という形ではなく,制度上想定された枠組みの中で一般利用が可能になっています.
\end{itemize}

\subsubsection*{引用}
\begin{itemize}
  \item 資料6.pdf(通信法規 p.61)「1.3 無線従事者でなくても行える無線設備の操作」および「電波法施行規則 第33条(簡易な操作)」「電波法 第27条の2(特定無線局の免許の特例)」の記載
  \item レジメ6.pdf(通信法規\_06 p.5)「電波法施行規則第33条(簡易な操作)」の条文要旨と「無線設備の操作の特例」の説明
  \item 資料3.pdf(通信法規 p.24)「特定無線局:適合表示無線設備のみを使用するもの」の注記,および「包括免許」手続の説明中の例(Ex. 携帯電話機)
  \item レジメ5.pdf(通信法規\_05 p.15)技術基準適合マーク(技適マーク)と「適合表示無線設備」の説明
\end{itemize}

\subsection{無線従事者の免許証を返納しなければならないのはどのような場合か?}
無線従事者の免許証は,次の場合に,総務大臣または総合通信局長へ返納しなければなりません.
\begin{itemize}
  \item 免許の取消しの処分を受けたとき(処分を受けた日から10日以内)
  \item 免許証の再交付を受けた後に,亡失した免許証を発見したとき(発見の日から10日以内)
  \item 無線従事者が死亡し,または失踪の宣告を受けたとき(戸籍法上の届出義務者が,遅滞なく当該免許証を返納)
\end{itemize}

\subsubsection*{引用}
\begin{itemize}
  \item 資料6.pdf 66頁「5.4 免許証の返納(無線従事者規則第51条)」
  \item レジメ6.pdf 12頁「免許の取消し/再交付後の発見/死亡・失踪の宣告時の返納」
\end{itemize}

\subsection{無線従事者と主任無線従事者の違いは何か?}
\begin{itemize}
  \item \textbf{無線従事者}:無線設備の操作(および操作の監督)を,法令で定める資格(操作範囲)の範囲内で行うことができる者である.無線設備の操作は原則として無線従事者が行う(ただし例外あり)。
  \item \textbf{主任無線従事者}:無線局において「無線設備の操作の監督を行う者」として\textbf{選任の届出がされた者}であり,その監督の下で,\textbf{無資格者や下位資格者}しかし,主任無線従事者が所持する資格の操作範囲内で,当該無線局の無線設備の操作を行えるようにする制度上の役割(ポスト)である.
\end{itemize}

したがって違いをまとめると,
\begin{enumerate}
  \item \textbf{性格の違い}:無線従事者は「資格を持って自ら操作(監督)できる立場」であるのに対し,主任無線従事者は「(届出により)当該無線局で他者の操作を監督する立場」である.
  \item \textbf{制度上の効果の違い}:主任無線従事者が選任・届出されている場合に限り,無資格者や下位資格者が,主任無線従事者の監督の下で操作できる(操作できる範囲は主任無線従事者の資格の操作範囲内)。
  \item \textbf{適用除外の違い}:モールス符号の送受を行う無線電信の操作など\textbf{特定の操作は主任無線従事者制度が適用されず},その場合は無線従事者でなければ操作できない.
  \item \textbf{要件の違い}:主任無線従事者として選任される者は,当該無線局に必要な操作範囲を含む無線従事者資格の所持に加え,所定期間内に指定講習機関が行う\textbf{主任講習}を受ける必要がある.
  \item \textbf{範囲の考え方}:無線従事者については,「操作できる範囲」と「監督できる範囲」は同一とされる.
\end{enumerate}

\subsubsection*{引用}
\begin{itemize}
  \item 資料6.pdf p.61「無線従事者でなければ行えない無線設備の操作(施行規則第34条の2)」:主任無線従事者制度の適用除外(モールス等)と,無線従事者でなければ操作できない旨
  \item 資料6.pdf p.63「無線従事者の操作及び監督の範囲(施行令第3条)」:無線従事者の操作範囲と監督範囲が同一である旨
  \item 資料6.pdf p.67「主任無線従事者制度(電波法第39条)」:免許人による主任無線従事者の選任届出,監督下での無資格者等の操作可,制度の適用されない場合の注意
  \item レジメ6.pdf(通信法規\_06)p.3:主任無線従事者の定義(選任の届出がされた監督者)と,主任制度の適用除外の具体例
  \item レジメ6.pdf(通信法規\_06)p.12:主任無線従事者制度の効果(無資格者・下位資格者の操作可)および主任講習の要件
\end{itemize}

\subsection{主任無線従事者制度について述べなさい.}
主任無線従事者制度とは,無線局(アマチュア無線局を除く.)の無線設備の操作について,本来は無線従事者による操作を原則としつつ,\textbf{無線従事者以外の者が操作を行う場合には,免許人が選任し,かつ選任の届出がされた「主任無線従事者」による監督を受けることにより操作を可能にする}制度である.

制度のポイントは次のとおりである.
\begin{itemize}
  \item \textbf{基本構造(監督による操作の許容)}\\
        無線従事者以外の者は,主任無線従事者として選任され,選任届出がされた者の監督を受けなければ,無線設備の操作(簡易な操作として省令で定めるものを除く.)を行ってはならない.
  \item \textbf{制度の採否と監督の範囲}\\
        主任無線従事者制度を採用するかどうかは免許人の選択による.主任無線従事者の監督の下で操作できる範囲は,主任無線従事者が所持する無線従事者資格の「操作・監督の範囲」によって画され(操作できる範囲と監督できる範囲は同一),無資格者や下位資格者ができる操作もその範囲内に限られる.
  \item \textbf{主任無線従事者に求められる要件(主任講習)}\\
        主任無線従事者として選任される者は,当該無線局に必要な操作範囲を含む無線従事者資格を所持していることに加え,所定の期間内に総務大臣が指定した講習機関(指定講習機関)が行う主任無線従事者の講習(主任講習)を受ける必要がある.指定講習機関は区分ごとに申請により指定され,指定がなされた場合,総務大臣は当該講習を行わない.
  \item \textbf{適用されない操作(主任無線従事者制度の例外)}\\
        モールス符号の送受を伴う無線電信の操作や,遭難通信・緊急通信・安全通信に関する一定の通信操作など,省令で定める特定の操作については主任無線従事者制度は適用されず,無線従事者でなければ操作できない.また,船舶・航空機が航行中で補充できない場合など,法令が定める場合には例外的に監督等の要件が緩和されることがある.
\end{itemize}

\subsubsection*{引用}
\begin{itemize}
  \item 資料6.pdf p.60(電波法第39条:主任無線従事者の定義,選任届出を受けた者の監督下での操作,アマチュア局除外,簡易操作除外,ただし書の例外)
  \item 資料6.pdf p.61(電波法施行規則第34条の2:主任無線従事者制度が適用されず無線従事者に限定される操作の例)
  \item 資料6.pdf p.63(電波法施行令第3条:無線従事者の操作範囲と監督範囲は同一)
  \item 資料6.pdf p.66(主任無線従事者制度の説明:採否は免許人の選択,主任講習の要否,制度適用外の例)
  \item 資料6.pdf p.67(指定講習機関の定義・指定の仕組み)
  \item レジメ6.pdf p.3(主任無線従事者の監督を受けなければ無資格者は操作できない旨)
  \item レジメ6.pdf p.12(主任無線従事者制度の概要:選任届出,監督下での無資格者等の操作可能範囲,主任講習)
\end{itemize}

\section{チェックポイント5}
\subsection{無線局の目的外使用の禁止とはどのようなことですか?}
無線局は,免許記録(免許の条件)に記載された事項に基づいて運用しなければなりません.したがって,
\begin{itemize}
  \item 免許記録に記載された「無線局の目的」,
  \item 「通信の相手方」および「通信事項(放送局の場合は放送事項)」,
\end{itemize}
の\textbf{範囲を超えて運用すること}(=免許された目的・相手方・通信事項以外に無線局を使うこと)が「目的外使用」であり,これを\textbf{してはならない}というのが「目的外使用の禁止」です.

なお例外として,人命・財貨の保全,社会の安寧・秩序の維持など\textbf{国民の福利に重大な関係をもつ通信}に限り,免許記録の範囲を超える運用が認められます.特に\textbf{遭難通信}を実施する場合は,規制が排除されるとされています.

\subsubsection*{引用}
\begin{itemize}
  \item 資料6.pdf p.68「第6節 運用/1 免許記録記載事項の遵守」(目的外使用の禁止と例外,遭難通信の扱い)
  \item 資料4.pdf p.36「免許記録(免許状)記載事項」(目的,通信の相手方及び通信事項等が免許記録に記載される旨)
\end{itemize}

\subsection{無線局の目的外の使用が許されるのはどのような場合ですか?}
無線局は,免許記録に記載された「目的」や「通信の相手方・通信事項(放送事項)」の範囲を超えて運用してはならず,これを「目的外使用の禁止」という.\\
ただし例外として,\textbf{人命・財貨の保全,社会の安寧,秩序の維持,その他国民の福利に重大な関係をもつ通信}に限り,免許記録記載事項の範囲を超えて運用することが認められる.\\
さらに,\textbf{遭難通信を実施する場合には,すべての規制が排除される}(目的外使用の禁止を含む各種規制より優先される)。

\subsubsection*{引用}
\begin{itemize}
  \item 資料6.pdf p.68「免許記録記載事項の遵守」内の「目的外使用の禁止」とその例外(人命・財貨の保全等に関する通信),および遭難通信時は規制排除の記述
\end{itemize}

\subsection{無線局を運用する場合において,できる限り擬似空中線を使用するよう義務付けられているのはどのような場合ですか?}
無線局は,次の場合に該当するときは,なるべく(できる限り)擬似空中線回路を使用しなければなりません.
\begin{itemize}
  \item 無線設備の機器の試験または調整するための運用をするとき
  \item 実験無線局を運用するとき
\end{itemize}
これは,試験・調整等の際に不要な電波を空中に放射して混信等を生じさせることを避ける趣旨です.

\subsubsection*{引用}
\begin{itemize}
  \item 資料6.pdf 69頁「3 擬似空中線回路の使用(電波法第57条)」
\end{itemize}

\subsection{無線局の運用における「暗語の使用」に関する電波法令の規定について知るところを述べなさい.}

電波法令では,無線局の運用に関して,特に「実験無線局」および「アマチュア局」について暗語の使用を禁止している.

\begin{itemize}
  \item 実験無線局およびアマチュア局は,暗語を使用してはならない(電波法第58条)。
  \item また,アマチュア局の送信装置は,通信に秘匿性を与える機能を有してはならない(無線設備規則第18条)。
  \item ここでいう暗語とは,第三者に通信内容を分からせないようにするため,通信当事者間で定めた方法により通信内容を秘匿するものをいう.
  \item さらに,「普通語の電信通信」に認められた実在語であっても,当該言語における通常の意味を有せず意味の分かる文章を構成しないもの,その他,普通語の条件を満たさない語辞・辞句等も暗語に該当し得る(普通語の条件を満たさないものとして整理される)。
\end{itemize}

\subsubsection*{引用}
\begin{itemize}
  \item 資料6「第6節 運用」p.69「4 暗語使用の禁止」:実験無線局・アマチュア局の暗語使用禁止(電波法第58条),アマチュア局送信装置の秘匿機能禁止(無線設備規則第18条),暗語の定義・具体的整理(普通語の条件を満たさない語等)
\end{itemize}

\subsection{無線通信の秘密の保護に関する電波法令の規定について知るところを述べなさい.}
電波法令における「無線通信の秘密の保護」は,憲法上の通信の秘密の保障を背景にしつつ,主として電波法第59条(通信の秘密の保護)と,これに違反した場合の罰則(電波法第109条等)によって確保されている.

\begin{itemize}
  \item \textbf{(基本原則:通信の秘密の尊重)}\\
        日本国憲法は,通信の秘密を侵してはならない旨を定めており,電波法における通信秘密保護の根拠・前提となる.

  \item \textbf{(電波法第59条:傍受・漏えい・窃用の禁止)}\\
        「何人も」,法律に別段の定めがある場合を除き,\textbf{特定の相手方に対して行われる無線通信}を\textbf{傍受}して,
        \begin{itemize}
          \item その\textbf{存在}または\textbf{内容を漏らすこと},
          \item これを\textbf{窃用すること}
        \end{itemize}
        をしてはならない.\\
        ここで保護対象となる通信は,\textbf{送信者と受信者が特定され,その間に特定性・個別性がある通信}である.

  \item \textbf{(「法律に別段の定めがある場合」:例示)}\\
        例として,犯罪捜査のための通信原書の押収(刑事訴訟法),電気通信の傍受する強制処分(刑事訴訟法),犯罪捜査のための通信傍受に関する法律などが挙げられる.したがって,電波法第59条は無条件の禁止ではなく,\textbf{法定の手続に基づく例外}を予定している.

  \item \textbf{(罰則:秘密漏えい・窃用に対する処罰)}\\
        電波法第59条の趣旨に反し,無線通信の秘密を漏えいまたは窃用する行為については罰則が設けられている.特に,
        \begin{itemize}
          \item \textbf{無線局の取扱中に係る無線通信の秘密保持違反}:\textbf{1年以下の拘禁刑または50万円以下の罰金}。
          \item \textbf{無線通信の業務に従事する者}が行う秘密の漏えいまたは窃用:\textbf{2年以下の拘禁刑または100万円以下の罰金}。
        \end{itemize}
        とされ,業務従事者にはより重い法的責任が課される.
\end{itemize}

\subsubsection*{引用}
\begin{itemize}
  \item レジメ6.pdf(通信法規\_06)p.17--18:電波法第59条の条文要旨,保護対象(特定性・個別性),禁止行為(漏えい・窃用・傍受)および「法律に別段の定めがある場合」の例示,日本国憲法第21条第2項の記載
  \item レジメ8.pdf p.15:通信の秘密に係わる違反の罰則(電波法第109条:無線局取扱中の秘密保持違反/無線通信業務従事者の秘密漏えい・窃用の刑)
  \item 資料8.pdf p.85:無線通信の秘密の漏洩または窃用(第109条第1項,第109条第2項)および刑罰の整理
\end{itemize}

\subsection{無線通信の原則について電波法令に規定するところを述べなさい.}
電波法令上,無線通信(無線局の運用)は,おおむね次の「免許(免許記録等)に基づく運用」を基本原則として規律されます.

\begin{itemize}
  \item \textbf{免許記録記載事項の遵守(目的外使用の禁止)}\\
        無線局は,免許記録に記載された「目的」または「通信の相手方」もしくは「通信事項(放送局の場合は放送事項)」の範囲を超えて運用してはならない.\\
        ただし,\textbf{遭難通信・緊急通信・安全通信・非常通信・放送の受信・その他総務省令で定める通信}については,この限りでない(目的外運用が許容される)。\\
        なお,遭難通信は,船舶または航空機が重大かつ急迫の危険に陥った場合に,遭難信号(SOS, MAYDAY等)を前置する方法などにより行う無線通信である.緊急通信は重大かつ急迫の危険に陥るおそれ等がある場合に緊急信号(XXX, PAN PAN等)を前置して行う無線通信であり,安全通信は航行に対する重大な危険を予防するため安全信号(TTT, SECURITE等)を前置して行う無線通信である.非常通信は,地震等の非常時に有線通信が利用できない(または著しく困難な)ときに,人命救助・災害救援・交通通信の確保・秩序維持のため行う無線通信である.\\
        また,「その他総務省令で定める通信」の例として,無線機器の試験または調整のために行う通信などが挙げられる.

  \item \textbf{運用時に遵守すべき無線設備の基本的同一性(設置場所・識別信号・型式・周波数)}\\
        無線局を運用する場合,\textbf{無線設備の設置場所,識別信号,電波の型式及び周波数}は,免許記録等に記載されたところによらなければならない.\\
        ただし,\textbf{遭難通信}についてはこの限りでない.

  \item \textbf{空中線電力の原則(範囲内・必要最小)}\\
        無線局を運用する場合,\textbf{空中線電力}は,
        \begin{itemize}
          \item 免許記録等に記載されたものの範囲内であること
          \item 通信するため必要最小のものであること
        \end{itemize}
        によらなければならない.\\
        ただし,\textbf{遭難通信}についてはこの限りでない.

  \item \textbf{(運用に付随する基本遵守事項)}\\
        無線局は,他の無線局等に運用を阻害するような混信その他妨害を与えないように運用しなければならない.\\
        また,無線設備の機器の試験または調整のための運用,実験無線局の運用の際は,なるべく擬似空中線回路を使用しなければならない.\\
        さらに,実験無線局・アマチュア局は暗語を使用してはならず,アマチュア局の送信装置は通信に秘匿性を与える機能を有してはならない.\\
        加えて,特定の相手方に対して行われる無線通信について,正当な根拠なく傍受してその存在もしくは内容を漏らし,またはこれを窃用してはならない(無線通信の秘密の保護)。
\end{itemize}

\subsubsection*{引用}
\begin{itemize}
  \item レジメ6.pdf(通信法規\_06)14--15ページ:目的外使用の禁止等(電波法第52条),設置場所等(第53条),空中線電力(第54条)
  \item 資料6.pdf 68--69ページ:免許記録記載事項の遵守(目的外使用の禁止の趣旨),遭難通信・緊急通信・安全通信・非常通信,放送の受信,その他総務省令で定める通信(施行規則第37条),混信等の防止(第56条),擬似空中線回路(第57条),暗語使用禁止(第58条)
  \item レジメ6.pdf(通信法規\_06)18ページ:無線通信の秘密の保護(電波法第59条),通信方法等(電波法第61条は無線局運用規則に委任)
\end{itemize}

\subsection{電波を発射する前にとるべき措置について電波法令に規定するところを述べなさい。}
電波の発射に際しては,事前に混信の発生を防止するための措置をとることが規定されている。

具体的には,無線局が相手局を呼び出そうとするときは,電波を発射する前に,
\begin{itemize}
  \item 受信機を最良の感度に調整すること
  \item 自局の発射しようとする電波の周波数その他必要と認める周波数によって聴守し,他の通信に混信を与えないことを確かめること
\end{itemize}
が必要である。

また,上記の聴守等の結果として他の通信に混信を与える虞があるときは,その通信が終了した後でなければ呼出しをしてはならない。

ただし,遭難通信・緊急通信・安全通信を行う場合,電波法第七十四条第一項に規定する通信を行う場合,および海上移動業務以外の業務において他の通信に混信を与えないことが確実である電波により通信を行う場合には,これらの措置を要しない。

\subsubsection*{引用}
\begin{itemize}
  \item 資料6.pdf 70ページ(PDF 11ページ目)「7.3 電波を発射する前の措置」および「無線局運用規則 第19条の2(発射前の措置)」第1項・第2項
\end{itemize}


\end{document}

