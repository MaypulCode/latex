% ■ アブストラクトの出力 ■
%	◆書式:
%		begin{jabstract}〜end{jabstract}	: 日本語のアブストラクト
%		begin{eabstract}〜end{eabstract}	: 英語のアブストラクト
%		※ 不要ならばコマンドごと消せば出力されない.


% 日本語のアブストラクト
\begin{jabstract}
	近年,自動運転レベル4(高度自動運転)以上の実現に向け,車両間通信(V2V)を用いた協調センシング技術が注目されている.
	協調センシングにより車両の死角情報を補完することで安全性が向上するが,高精細なセンサデータをリアルタイムに共有するためには,ギガビット級の伝送速度とミリ秒オーダの低遅延性を両立する通信システムが不可欠である.
	ミリ波通信(FR2)はこの要求を満たす有望な技術であるが,電波の直進性が強く,他車両による動的遮蔽が頻発する道路環境では通信品質が急激に劣化し,通信断が発生しやすいという重大な課題がある.

	通信断を防ぐため,遮蔽に強いSub-6GHz帯(FR1)を併用するマルチバンド通信が検討されているが,既存の制御手法には一長一短がある.
	リアクティブ制御では,通信品質劣化後に切り替えるため切替処理の遅延によりパケット損失が生じる.
	一方,従来の予測型制御では,将来の劣化を予測して事前に切り替えることでパケット損失を防げるものの,一時的なフェージングに対しても過剰に反応してしまい,不要な帯域切替(ピンポン効果)により周波数利用効率が低下するという問題がある.
	したがって,動的遮蔽環境下において,高い信頼性と周波数利用効率を両立させる新たな制御手法が求められている.

	本研究では,物理層のSINR(Signal-to-Interference-plus-Noise Ratio)予測とMAC層のHARQ(Hybrid Automatic Repeat reQuest)を連携させた,適応的帯域切替手法を提案する.
	提案手法は,物理層におけるSINRの二重指数平滑法(Holt法)によるトレンド予測と,適応的EWMAを用いて突発的な急激変化を監視する二段階の監視アルゴリズムを採用し,これをMAC層のHARQ制御と密接に連携させた.
	これにより,一時的なフェージングによる品質劣化と,動的遮蔽による持続的な品質劣化を正確に識別する.
	一時的な劣化に対してはHARQの再送機能を用いて回復を図り,持続的な遮蔽が予測される場合にのみ,事前にSub-6GHz帯へハンドオーバを行う.
	さらに,制御パラメータの決定にはOptunaを用いたベイズ最適化を適用し,環境に応じた最適なチューニングを実現した.
	このアプローチにより,無駄な帯域消費を抑えつつ,通信断を未然に防ぐことが可能となる.

	提案手法の有効性を検証するため,ネットワークシミュレータOMNeT++と交通流シミュレータSUMOに,3GPP TR 38.901準拠のチャネルモデルを実装した統合シミュレーション環境を構築した.
	現実的な高速道路環境における動的遮蔽シナリオでの評価の結果,提案手法は\SI{96.40}{\percent}のパケット配信率(PDR)を達成した.
	これは,ミリ波のみのベースライン手法(\SI{59.58}{\percent})に対し約\SI{37}{\percent},HARQのみの手法(\SI{90.33}{\percent})に対し約\SI{6}{\percent}の大幅な改善である.
	また,平均遅延は\SI{4.56}{ms}に抑えられ,V2Xサービスの要求条件(\SI{10}{ms}以下)を十分に満たした.
	さらに,スループットの累積分布関数(CDF)における5パーセンタイル値の向上を確認し,最悪ケースにおいても安定した通信が可能であることを示した.
	加えて,遮蔽発生時においてもFR1による約\SI{30}{Mbps}のスループット維持を実現した.
	これらの結果より,提案手法は周波数利用効率を維持しながら,動的遮蔽環境下においても高い信頼性を実現できることを実証した.
\end{jabstract}
